\documentclass[11pt, oneside]{article}   	
\usepackage{geometry}                		
\geometry{a4paper, total={7in, 9in}, margin=0.4in}                   		
\usepackage{graphicx}				
\usepackage{amssymb}
\usepackage{listings}
\usepackage{longtable}
\usepackage{tabularx}

\parindent=0pt

\title{Project Milestone 2 part A}
\author{Abhi Agarwal}
\date{}

\begin{document}
\maketitle

\textwidth = 540pt

\par My project two documentation is organized in a way where I construct the table in the beginning, and then break down the details I'm going to present and explain them in detail (for documentation and understanding). I have also attached definitions of word I'm using at the bottom to reduce ambiguity and misunderstandings. I tried to simplify my table by using function calls to test the different types, and that's something I think is a limitation of my Syntax Directed Definition, but I think it was important for me to do that. 

\par I have an understanding of how to implement these functions in Hacs. My apologies I took a few shortcuts, but I had to re-write most of my SDD and then was pressed for time. I couldn't complete the whole SDD in time. The parts which I think are a little weak: 2.2, 2.3, 2.4.1, 2.4.3, 2.4.11, 2.4.12, and my general scoping. I understand these issues.

\section*{Syntax Directed Definition Table}

\begin{tabularx}{\textwidth}{ |X|X| }
\hline
\textbf{Production} & \textbf{Semantic Rules} \\
  
% A Program declaration
\hline $AProg \rightarrow ABlock_1$  &  ABlock.inScope = top; $ABlock_1.s$ = \{\}; \\

% A Block declaration:: Can be either a class, function or variable
\hline $ABlock \rightarrow C_1; \, ABlock_1$  &  $C_1.s = ABlock.s; ABlock_1.s = C_1.s$ \\
	   $| \, F_1; \, ABlock_1$ & $F_1.s = ABlock.s; ABlock_1.s = F_1.s$ \\
	   $| \, \epsilon$ & \\
	   
% A Class declaration
\hline $C \rightarrow class \, id \, \{CS_1\} $ & $ C.s \, $=$ \, Extend(C.s, id.sym, class); \newline C.inScope = class; CS_1.s = C.s; $ \\

% A Function declaration
\hline $F \rightarrow function \, T_1 \, id(P_2) \, \{S_3\}$ & $F.s \, $=$ \, Extend(F.s, id.sym, function); \newline F.inScope = function; S_3.s = F.s; $ \\

% A Method Member declaration
\hline $M \rightarrow T_1 \, id(P_2) \, \{S_3\} $ & $ M.s \, $=$ \, Extend(M.s, id.sym, method); \newline M.inScope = method; S_3.s = M.s; $\\

% A Class Statement declaration
\hline $CS \rightarrow S_1; CS_2 $ & $ S_1.s = CS.s; CS_2.s = S_1.s $\\
	   $| \, M_1; CS_2$ & $ M_1.s = CS.s; CS_2.s = M_1.s $ \\
	   $| \, \epsilon$ & \\

% Statement declarations:: var x; if; while;
\hline $S \rightarrow T_1 \, id; $ & $T_1.s \, $=$ \, S.s; \, S.s \, $=$ \, Extend(S.s, id.sym, T_1.type)$ \\
	   $| \, if(E_1) \, \{S_2\} $ & $ E_1.s = S.s; \, $checkIfBoolean$ \, (E_1.type);  S_2.s = S.s $ \\
	   $| \, while(E_1) \, \{S_2\} $ & $ E_1.s = S.s; \, $checkIfBoolean$ \, (E_1.type);  S_2.s = S.s $ \\

% Parameter Check declaration
\hline $P \rightarrow T_1 \, id, P_2; $ & $T_1.s \, $=$ \, P.s; \, P.s \, $=$ \, Extend(P.s, id.sym, T_1.type); \newline P_2.s = P.s; $ \\
          $| \,  T_1 \, id; $ & $T_1.s \, $=$ \, P.s; \, P.s \, $=$ \, Extend(P.s, id.sym, T_1.type) $ \\
	   
% Expression declaration
\hline $E \rightarrow this.E_1; $ & $ classScope(E.inScope); E_1.s = E.s $ \\

	   $| \, E_1.T_2 $ & $ E_1.s = E.s; T_2 = E.s; checkIfObject(E_1.type); \newline checkIfString(T_2.type); \newline classContainsAttribute(E_1, T_2) $ \\
	   
	   $| \, !E_1 $ & $ E_1.s\,$=$\,E.s; checkIfBoolean(E_1.type); \newline E.type\,$=$\,E_1.type; $ \\
	   
	   $| \, -E_1 $ & $ E_1.s = E.s; checkIfInt(E_1.type); E.type = E_1.type; $ \\
	   $| \, +E_1 $ & $ E_1.s = E.s; checkIfInt(E_1.type); E.type = E_1.type; $ \\
	   $| \, -E_1 $ & $ E_1.s = E.s; checkIfInt(E_1.type); E.type = E_1.type; $ \\
	   
	   $| \, E_1*E_2 $ & $ E_1.s = E.s; E_2.s = E.s; \newline checkIfInt(E_1.type, E_2.type); E.type = E_1.type; $ \\	   
	   $| \, E_1/E_2 $ & $ E_1.s = E.s; E_2.s = E.s; \newline checkIfInt(E_1.type, E_2.type); E.type = E_1.type; $ \\	
	   $| \, E_1-E_2 $ & $ E_1.s = E.s; E_2.s = E.s; \newline checkIfInt(E_1.type, E_2.type); E.type = E_1.type; $ \\
	   $| \, E_1\%E_2 $ & $ E_1.s = E.s; E_2.s = E.s; \newline checkIfInt(E_1.type, E_2.type); E.type = E_1.type; $ \\ 
	   
	   % Return is int or string
	   $| \, E_1+E_2 $ & $E_1.s \, $=$\, E.s; E_2.s = E.s; \, \newline checkSameType(E_1.type, E_2.type); E.type = E_1.type; $ \\	
	   % Return is boolean and same type
	   $| \, E_1<E_2 $ & $E_1.s \, $=$\, E.s; E_2.s = E.s; \, \newline checkSameType(E_1.type, E_2.type); \newline checkIntOrString(E_1.type, E_2.type); E.type\,$=$\,boolean; $ \\
	   $| \, E_1<=E_2 $ & $E_1.s \, $=$\, E.s; E_2.s = E.s; \, \newline checkSameType(E_1.type, E_2.type); \newline checkIntOrString(E_1.type, E_2.type); E.type\,$=$\,boolean; $ \\
	   $| \, E_1>E_2 $ & $E_1.s \, $=$\, E.s; E_2.s = E.s; \, \newline checkSameType(E_1.type, E_2.type); \newline checkIntOrString(E_1.type, E_2.type); E.type\,$=$\,boolean; $ \\
	   $| \, E_1>=E_2 $ & $E_1.s \, $=$\, E.s; E_2.s = E.s; \, \newline checkSameType(E_1.type, E_2.type); \newline checkIntOrString(E_1.type, E_2.type); E.type\,$=$\,boolean; $ \\
	   	   
\hline
\end{tabularx}	   
	   
\begin{tabularx}{\textwidth}{ |X|X| }
\hline
\textbf{Production} & \textbf{Semantic Rules} \\
	   
\hline
	   
	   $| \, E_1==E_2 $ & $E_1.s \, $=$\, E.s; E_2.s = E.s; \, \newline checkSameType(E_1.type, E_2.type); E.type = boolean; $ \\
          $| \, E_1!=E_2 $ & $E_1.s \, $=$\, E.s; E_2.s = E.s; \, \newline checkSameType(E_1.type, E_2.type); E.type = boolean; $ \\
	   $| \, E_1\&\&E_2 $ & $E_1.s \, $=$\, E.s; E_2.s = E.s; \, \newline checkifBoolean(E_1.type, E_2.type); E.type = boolean; $ \\
	   $| \, E_1 \| E_2 $ & $E_1.s \, $=$\, E.s; E_2.s = E.s; \, \newline checkifBoolean(E_1.type, E_2.type); E.type = boolean; $ \\
	   
	   $| \, E_1 += E_2 $ & $ E_1 = E.s; E_2.s = E.s;  $ \\
	   
	   $| \, E_1.length $ & $ E_1.s = E.s; checkIfString(E_1.type); \newline E.type = E_1.type $ \\
	   $| \, E_1.charCodeAt(T_2 id) $ & $ E_1.s = E.s; checkIfString(E_1.type); T_2.s = E_1.s; \newline checkIfInt(T_2.type); E.type = E_1.type $ \\
	   $| \, E_1.charCodeAt(T_2 id, T_3 id) $ & $ E_1.s = E.s; checkIfString(E_1.type); \newline T_2.s = E_1.s; T_3 = E_1.s; \newline checkIfInt(T_2.type, T_3.type); E.type = E_1.type $ \\
	   
	  % Generic return
	   $| \, L_1 $ & $ L_1.s = E.s; L.type = L_1.type; $ \\	  	   
	   
% Object Literal decleration
\hline $OL \rightarrow id:T_1, OL_2 $ & $ T_1.s = OL.s; Extend(OL.s, id.sym, L_1.type); \newline OL_2.s = OL.s $ \\

% Literal declaration
\hline $L \rightarrow id$ & $L.type = Lookup(L.s, id.sym)$ \\

% Types
\hline $T \rightarrow integer$ & $T.type = int$ \\
  	   $| \, string$ & $T.type = string$ \\  
  	   $| \, class$ & $T.type = class$ \\  
  	   $| \, function$ & $T.type = function$ \\  

\hline
\end{tabularx}

\section*{1 - Limitations and Decisions}
\begin{enumerate}
\item Inherited Attributes: s, Synthesized Attributes: inScope, Functions: checkifBoolean, Extend, Lookup, classScope, checkIfInt, checkIfIntOrBoolean, checkSameType
\item Declarations: \textbf{AProg} is the beginning of the program, \textbf{ABlock} is a block (either a function or a class), \textbf{C} is a class, \textbf{F} is a function, \textbf{M} is a member function, \textbf{CS} is a class statement (either a statement or a member function), \textbf{S} is a statement, \textbf{E} is an expression, \textbf{L} is a literal, \textbf{T} is a type, \textbf{OL} is an object literal definition, and \textbf{P} is a parameter check.
\item The shortcuts I took for this assignment: classContainsAttribute
\item My way to know the scope was to pass down a synthesized attribute called inScope.
\item The block is each module of the particular program where module could be function, or a class.
\item For this particular implementation there is no support for global variables, but I have thought about it. It would be a part of the modules that I have described above, and I would extend the modules to be: function, class, or global variables.
\item  For an object declaration \texttt{var} is necessary as suggested in the example.
\item Two passes through the code: first to declare and get all the information, and second to validate it.
\item Declaration of many functions to check if they are the correct type. They are straight forward, and do what the function names are. They are very similar to the safeCast() function we created for HW6. For example: checkIfBoolean(...) checks if a particular type is boolean - if it is then return Error();
\item My implementation I extend a class declaration into the Class scope. This is because I'm now able to check if I'm inside a class or not by looking through the immediate scoped variables.
\item My implementation of the string and its additional operators are included within the Expressions (E).
\item My understanding was that class and function were also types. I think this is correct in my implementation because I Lookup functions and classes as types.
\item With my design there shouldn't be an expression with a comma in between because I have a special Production for things within the parentheses. 

\section*{2 - Extra notes}

\item To me (my assumption is that) \texttt{any} means that we're able to convert between types very explicitly without using a cast, and we're able to assign certain values regardless of the type. Moreover, the every type is assignment compatible with itself is fairly straight forward, and the standard case. 
\item Identifier literals: must exist within the scope where it is declared, and has to have the correct type. We have to do a type check to see if the assignment is the same as the type it was declared as.
\item Int literals: We've to check if they've the type int to classify them as Integers. This is just mechanical as we've to simply check if their type matches the type \texttt{int}.
\item Object literals: String literals: We've to check if they've the type string to classify them as  \texttt{string}.
\item Object literals: We must check if $m_k$ for each $k$ to see if it is not already within the current scope, and if it is then we have to generate an error as they must be distinct.
\item Object literals: We also have to check from the scope of the class to make sure the $m_k$ exists in the class type.
\item l-value: We also have to check for the type of $l_k$, for a given $k$, and if it has a type which is assignment compatible to the type declared for the member then it's been validated.
\item l-value: We have to check whether the lexeme on the left site is either an Identifier or an instance of a class variable for which the element it is trying to access exists, and if either of these conditions are not present then we return an error.
\item l-value: We have to check whether the lexeme on the left site is either an Identifier or an instance of a class variable for which the element it is trying to access exists, and if either of these conditions are not present then we return an error.
\item Expressions: We have to check if the scope we're in is in fact a class block, and if it is then we have to check if the variable has been defined in the local nested scope. If the variable hasn't been defined then we throw an error because it can't be referenced.
\item this: We have to check if the scope we're in is in fact a class block, and if it is then we have to check if the variable has been defined in the local nested scope. If the variable hasn't been defined then we throw an error because it can't be referenced.
\end{enumerate}

\section*{Definitions}
\par Skip over this section, but here I've defined some things I think are important, and I wanted to have the definitions here to reduce ambiguity and misunderstandings. 
\par \textbf{Scope}: Tied to program ``blocks" (lexical) vs runtime stack.
\par \textbf{Global scope}: Scope that surrounds a block where the block can be a function, variable, object, or class.
\par \textbf{Top scope}: The scope of the whole program that includes all declarations of classes, functions, and global variables that are declared in the main script.
\par \textbf{Current scope}: For a given block we're currently in, all the local functions/variables/classes we've encountered.
\par \textbf{New scope}: We're going into a new program block, and therefore we declare a new nested scope including values from the global scope.

\end{document}  